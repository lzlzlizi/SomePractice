\section{System Overview}

\emph{\normalsize TopicStream} is designed to track and understand the dynamic characteristics of text streams.
%Accordingly, it consists of two major modules: streaming tree cut and streaming visualization (Fig.~\ref{fig:overview}).
\kg{It} consists of two major modules: streaming tree cut and streaming visualization (Fig.~\ref{fig:overview}).\looseness=-1

%The input of the streaming tree cut is a set of topic trees with tree cuts as well as a set of new coming documents.
The input of the streaming tree cut is a set of topic trees with tree cuts \kg{and} a set of \kg{incoming} documents.
In \emph{\normalsize TopicStream}, the topic trees are derived by the evolutionary tree clustering method developed by Wang et al.~\cite{Wang2013}.
The basic idea of this method is to balance the fitness of a tree and the smoothness between trees by a Bayesian online filtering process.
We derive the tree cuts based on the user-selected focus node(s).
%This module first extracts a topic tree from the newly arrived documents by the evolutionary tree clustering model~\cite{Wang2013}.
This module \kg{initially} extracts a topic tree from the newly arrived documents \kg{using} the evolutionary tree clustering model~\cite{Wang2013}.
%Then a tree cut is derived from the new topic tree via the developed streaming tree cut algorithm.
\kg{A} tree cut is \kg{then} derived from the new topic tree \kg{through} the developed streaming tree cut algorithm.\looseness=-1

%Next, the streaming tree cuts are fed to the visualization module.
The streaming tree cuts are \kg{then} fed into the visualization module.
%We employ the visual sedimentation metaphor to reveal the merging process of newly arrived documents into the dominant center of the visualization, the river part.
We employ the visual sedimentation metaphor to reveal the merging process of newly arrived documents \kg{with} the dominant center of visualization.
% \kg{which is} the river part .
The circle packing algorithm is also developed to illustrate the relationships of document clusters within each topic stripe, including their similarity and temporal relationships~\cite{Wang2006visualization,ZhaoTVCG2014}.\looseness=-1
