% !TEX root = EvoTree-KDD.tex

\section{Potential Applications}\label{sec:application}

We deployed the system as an internal desktop application within Microsoft. A panel of experts, including customer relationship managers, public relations managers, and researchers, systematically evaluated and investigated the potential applications of the system.  Several highly promising applications were identified.

%We invited several experts, including 3 customer relationship managers, 3 public relations managers, and 4 researchers, to systematically evaluate and discuss the potential usage of the system.


First, our system can be employed in customer relationship management to help analysts fully understand customer-related information on social media.
A variety of social media outlets enable customers to communicate with each other more easily and take a more active role as market players~\cite{Hennig2010}.
%Since the buzz of the crowd on social media provides a great deal of information that
%was not available before, businesses have started leveraging social media to profile customers, derive brand perception, understand buying trends, and improve customer satisfaction.
The implication for businesses is obvious: they need to understand what customers are saying about them in various social media outlets.
For this reason, customer relationship managers are very interested in using the system to quickly understand customers' comments and feedbacks on social media.
For example, a customer relationship manager utilized the system to analyze customer feedback related to ``Xbox'' on Twitter and blogs.
He appreciated the topic hierarchies provided by the system, which enabled him to immediately identify overall patterns as well as directly interact with the visualization to examine the finer-grained topics.
He commented, ``This is very helpful in finding something unexpected.''

Second, public relations managers claimed that the system can help them better analyze company-related information, such as news, blogs, and microblogs.
For example, a Microsoft public relations analyst wants to analyze a number of important events related to the company that occurred in the first half of 2013.
These events are the announcement of the new Xbox, Motorola suing the company, and the release of Windows 8.
To analyze these events, the public relations analyst employed the system to examine the Microsoft-related news stream and understand the major topics and how they have changed over time.
With the system, the analyst could better understand how these events are related to one another at different granularities and their impact over time.
Furthermore, he can determine whether the company's public relations strategies have succeeded (e.g., the amount of the excitement generated by the product release).

Finally, researchers utilized our system to study major research topics in their fields as well as the splitting/merging relationships among the topics of interest over time.
With this toolkit, they easily identified the research topics and related publications that match their research interest.
Interestingly, a sociology PhD student found that he could use the system to study the media framing effects in a news corpus.
He was intrigued when he found that a sub-topic changed its parent between two adjacent trees.
In his exploration, the sub-topic exemplified by the keywords ``Windows, Kinect" changed its parent from ``Windows" to ``Xbox."
He said, ``Parent topics provide the context for the sub-topics.
A concrete sub-topic such as ``Kinect'' can be mentioned in the context of ``Windows'' in the first week, and in the context of ``Xbox'' in the next week.
This defines different perspectives or ways of communicating about topics.
This phenomenon is widely studied in our field and is referred to as \textbf{media framing}.
I believe this system can definitely help detect such framing phenomena." 